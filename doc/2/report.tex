\documentclass[a4paper, 11pt]{article}
\usepackage{graphicx}
\usepackage{amsmath}
\usepackage[pdftex]{hyperref}
\usepackage[a4paper]{geometry}
\usepackage{hyperref}

% Lengths and indenting
\setlength{\textwidth}{16.5cm}
\setlength{\marginparwidth}{1.5cm}
\setlength{\parindent}{0cm}
\setlength{\parskip}{0.15cm}
\setlength{\textheight}{22cm}
\setlength{\oddsidemargin}{0cm}
\setlength{\evensidemargin}{\oddsidemargin}
\setlength{\topmargin}{0cm}
\setlength{\headheight}{0cm}
\setlength{\headsep}{0cm}

\renewcommand{\familydefault}{\sfdefault}

\title{Machine Learning 2013: Project 2 - SVM Report}
\author{anufer@student.ethz.ch\\ elmerl@student.ethz.ch\\ nivo@student.ethz.ch\\}
\date{\today}

\begin{document}
\maketitle

\section*{Experimental Protocol}
Usage:\\
Download the csv files to /data/....csv (... = training, testing, validation)\\
Run svm.m \\
Results are in /data/....out (... = training, testing, validation)

\section{Tools}

\begin{itemize}
\item Matlab (code is in /code/ directory)
\item Git / \href{https://github.com/lukaselmer/ethz-machine-learning}{Github Repository} \footnote{https://github.com/lukaselmer/ethz-machine-learning}
  %\begin{itemize}
    % item \href{https://github.com/lukaselmer/ethz-machine-learning/releases/tag/best\_v3}{Final solution} \footnote{https://github.com/lukaselmer/ethz-machine-learning/releases/tag/best\_v3} (tag best\_v3, master branch, code attached in the zip file) 
  %\end{itemize}
\end{itemize}

\section{Algorithm}
\label{sec:Algorithm}
For training the Support Vector Machine we used the \textit{svmtrain}\footnote{http://www.mathworks.ch/ch/help/stats/svmtrain.html} function from the Statistics Toolbox of Matlab. And for classification of new data we used \textit{svmclassify}\footnote{http://www.mathworks.ch/ch/help/stats/svmclassify.html}, which is also a function of the Statistics Toolbox.

These two built-in function form the core of the algorithm. To improve the results, we added an additional pre process step to alter the input data before we give them to \textit{svmtrain}.
\section{Features}
\label{sec:Features}


\section{Parameters}
\label{sec:Parameters}
The support vector machine with the Gaussian Radial Basis Function as kernel, takes 2 parameters: The box constraint C for the soft margin, and a scaling factor $\sigma$.\\
To find the optimal values for these parameters we first used Grid search, to narrow down the range to $\sigma \in [0.5, 0.6]$ and $C \in [1, 1.2]$.\\
Second, we randomly search in these ranges for the best values which resulted in following values:\\
\begin{itemize}
\item $\sigma=0.556201641$
\item $C=1.316157273$
\end{itemize}

\section{Lessons Learned}


\end{document} 
