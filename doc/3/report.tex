\documentclass[a4paper, 11pt]{article}
\usepackage{graphicx}
\usepackage{amsmath}
\usepackage[pdftex]{hyperref}
\usepackage[a4paper]{geometry}
\usepackage{hyperref}

% Lengths and indenting
\setlength{\textwidth}{16.5cm}
\setlength{\marginparwidth}{1.5cm}
\setlength{\parindent}{0cm}
\setlength{\parskip}{0.15cm}
\setlength{\textheight}{22cm}
\setlength{\oddsidemargin}{0cm}
\setlength{\evensidemargin}{\oddsidemargin}
\setlength{\topmargin}{0cm}
\setlength{\headheight}{0cm}
\setlength{\headsep}{0cm}

\renewcommand{\familydefault}{\sfdefault}

\title{Machine Learning 2013: Project 3 - Text Classification Report}
\author{anufer@student.ethz.ch\\ elmerl@student.ethz.ch\\ nivo@student.ethz.ch\\}
\date{\today}

\begin{document}
\maketitle

\section*{Experimental Protocol}
Usage:\\
Download the csv files to /data/3/....csv (... = training, testing, validation)\\
Run map.m \\
Results are in /data/3/....out (... = training, testing, validation)










\section{Tools}

\begin{itemize}
\item C\#, LINQ, Visual Studio 2012 Ultimate (code is in /code/ directory)
\item Matlab (code is in /code/ directory)
\item Git / \href{https://github.com/lukaselmer/ethz-machine-learning}{Github Repository} \footnote{https://github.com/lukaselmer/ethz-machine-learning}
  %\begin{itemize}
    % item \href{https://github.com/lukaselmer/ethz-machine-learning/releases/tag/best\_v3}{Final solution} \footnote{https://github.com/lukaselmer/ethz-machine-learning/releases/tag/best\_v3} (tag best\_v3, master branch, code attached in the zip file) 
  %\end{itemize}
\end{itemize}

\section{Algorithm}

Principal component analysis

Describe the algorithm you used for classification.

\section{Features}
To group similar words together, reprocessing was used. First, we used the Levinsthein distance\footnote{https://en.wikipedia.org/wiki/Levenshtein\_distance}, but then switched to a slightly modified version of the edit distance\footnote{https://en.wikipedia.org/wiki/Edit\_distance}, which gave slightly better results. Also, instead of using an absolute value threshold for the distance, we used a ratio of about 75\% to group similar words together. For this preprocessing, we used C\#, because string handling in C\# seemed much easier than in Matlab.\\

After the preprocessing, we then used Matlab to predict the city codes and country codes. For this, we used PCA.

\section{Parameters}
To find the parameters, we used manual testing. The most important feature seemed to be the edit distance ratio, which is 75\%. Another important parameter is the amount of top words which are picked at the start of the algorithm. These are the words, which are in a category of it's own before grouping them together.

\section{Lessons Learned}
Many other tools and algorithms have been tried. However, one we didn't try and might have worked well is SVM.

\end{document} 
